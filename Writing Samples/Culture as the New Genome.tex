\documentclass[11pt,a4paper]{article}

\usepackage[hidelinks]{hyperref}
\usepackage[usenames, dvipsnames]{color}
\usepackage[object=vectorian]{pgfornament}
\usepackage{url}
\urlstyle{same}

\usepackage[quiet]{fontspec}
\defaultfontfeatures{Mapping=tex-text}
\setmainfont[Ligatures=TeX]{Sabon LT Std}

\usepackage{lettrine}

\definecolor{spot}{rgb}{0.6,0,0}

\usepackage[margin=1in]{geometry}

\usepackage[superscript,biblabel]{cite}
%\usepackage[square]{natbib}
%\setcitestyle{super}

\title{\Large{Writing Sample 3}\\\bigskip \vspace{0.1cm}\huge{\color{spot}Culture as the New Genome}\vspace{0cm}}
\author{Milind Hegde}
\date{}

%\pagestyle{empty}

\begin{document}
\fontsize{11.25}{18}\selectfont
\setlength{\parskip}{0.15cm plus4mm minus3mm}
\maketitle
%\thispagestyle{empty}

\begin{center}
\textcolor{spot}{\rule[2cm]{\textwidth}{1.5pt}}
\vspace*{-2.4cm}
\end{center}

\lettrine[lines=2, findent = 1pt, nindent = 3pt, loversize=0.05]{I}{} believe that culture can be called the new genome. But what is meant by the statement, ``Culture is the new genome’’? What exactly is culture? The interpretation used here is that culture refers to art, the appreciation of beauty, as well as the more abstract form of humans working together, developing language, developing new tools, showing creativity in their actions, etc. Because of its magnificent power and the advances it has brought, culture has replaced the well-known biological genome in importance in humans for several thousand years---and this makes it worthy of greater study, even ignoring its inherent fascinations.

Culture is largely responsible for humanity’s great advances in the past 10,000 years or so, which has catapulted us well-beyond the successes of any other species. It allowed us to learn new skills develop new technologies over the course of mere centuries or decades instead of the hundreds of thousands of years it may have taken to evolve the same capabilities though pure natural selection, an unprecedented evolutionary advantage.

Understanding culture, including its biological origins, its home in the brain, its effects on the brain, and how we can perhaps recreate it in an artificial intelligence is a subject that requires deep input from many, many fields. Richard Dawkins was perhaps the first to popularize the idea of humankind adopting culture as a new genome, i.e. a new mode of rapid evolution, in his book \emph{The Selfish Gene}\cite{dawkins}. He proposed the idea of memes as being the fundamental units of cultural expression and transmission through our species, similar in function and scope to genes. And like genes, memes too replicate and compete for the resources of time and attention of our minds, giving rise to a selective force which can cause evolution. As Dawkins recognizes, the advent of memes and cultural evolution is what made the crucial difference between humans and our other primate cousins.

Another aspect of culture from a biological point of view is its actual mechanism of function in the human brain. Tantalizing clues have been found in neuroscience, with the most important recent development of the discovery of mirror neurons, which promise to be the neurological basis for our ability to empathize with others. By firing the same way whenever we see another person performing an action as it would had we done the action ourselves, they, in a somewhat literal sense, simulate others’ points of view within our own heads. This is an astonishing discovery whose consequences will continue to be explored for years to come.

The existence of culture creates an intriguing question for artificial intelligence. Can a computer be called truly intelligent if it is unable to comprehend, participate in, and create culture on its own? To what extent can a human created intelligence produce thoughts and ideas that are truly its own and not just an amalgamation of numerous pre-programmed artefacts? Is it enough for a computer to be able to tell us who Isaac Newton is, or is there something more to ``knowing’’ about Newton? (Noam Chomsky is strongly of the latter opinion.\cite{chomsky}) Another extremely relevant issue is that of creativity of artificial intelligences, for which the problem is that it is unclear what a definition of creativity would resemble in the first place.  For example, is it just a monumentally complex process, or is there an element of true randomness and irrationality which would make it impossible to accurately capture in a computer system? And if a machine truly were creative, would we be in a position to recognize it? (An interesting example of this was the AARON project \cite{aaron}.)

It is clear that there is an abundance of things and ideas to explore in this subject. The areas of cognitive science, neuroscience and artificial intelligence are young and still open to new ideas and problems. As George Miller\cite{miller} noted over forty years ago, these fields are connected and will remain so, gaining insights and approaches from each other. One example of a problem where these three fields are all involved is that of art and aesthetics. Cognitive science is interested in understanding the mental processes that recognize and create art, while neuroscience is interested in the neuron-level mechanisms involved in recognizing and appreciating art and beauty of all forms. Artificial intelligence, of course, attempts to implement, in a sense, our knowledge of the human perception of beauty into computer systems---initially as mimicry, but perhaps later as a deeper, emergent property of the system, in a manner similar to how it presumable emerges in the human brain.

It is not true that mirror neurons or culture are solely responsible for humans’ rapid development in the past few thousand years, since they do not address the source of new ideas and innovations, only methods of their spread. However, it is true that their absence would have made it impossible for us to reach our current position---if there is no one to learn your idea, it will soon die and be forgotten. The same is true for our traditional genes---they do not address how the winning combination of traits come about, but when they do, genes and natural selection will ensure that they stick around. Culture is the reason we are human, and, in a self-referential twist, it is what allows us to explore the idea of culture in the first place. For this reason culture is truly the new genome.



\vspace{0.75cm}


\begin{center}
\pgfornament[width=7.5cm, color=spot]{85}
\end{center}

\newpage
\color{spot}
\begin{thebibliography}{99}
\color{black}

\bibitem{dawkins} Richard Dawkins. \emph{The Selfish Gene}.

\bibitem{chomsky} The Atlantic:
\emph{Noam Chomsky On Where Artificial Intelligence Went Wrong}. Last accessed September 20, 2013. URL: \url{http://www.theatlantic.com/technology/archive/2012/11/noam-chomsky-on-where-artificial-intelligence-went-wrong/261637}.

\bibitem{aaron}Harold Cohen. \emph{The Further Exploits of AARON}. Last accessed September 20, 2013. URL: \url{http://www.stanford.edu/group/SHR/4-2/text/cohen.html}

\bibitem{miller}George Miller. \emph{The Cognitive Revolution: A Historical Perspective}.

\bibitem{ramachandran} V.S. Ramachandran. \emph{The Tell-Tale Brain}.

\end{thebibliography}
\end{document}