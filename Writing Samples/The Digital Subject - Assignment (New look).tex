\documentclass[11pt,a4paper]{article}

\usepackage[hidelinks]{hyperref}
\usepackage[usenames, dvipsnames]{color}
\usepackage[object=vectorian]{pgfornament}
\usepackage{url}
\urlstyle{same}

\usepackage[quiet]{fontspec}
\defaultfontfeatures{Mapping=tex-text}
\setmainfont[Ligatures=TeX]{Sabon LT Std}

\usepackage{lettrine}

\definecolor{spot}{rgb}{0.6,0,0}

\usepackage[margin=1in]{geometry}

\usepackage[superscript,biblabel]{cite}
%\usepackage[square]{natbib}
%\setcitestyle{super}

\title{\Large{Writing Sample 1}\\\bigskip \vspace{0.1cm}\huge{\color{spot}Digital Intimacy}\vspace{-0.2cm}}
\author{Milind Hegde}
\date{}

%\pagestyle{empty}

\begin{document}
\fontsize{11.25}{18}\selectfont
\setlength{\parskip}{0.15cm plus4mm minus3mm}
\maketitle
%\thispagestyle{empty}

\begin{center}
\textcolor{spot}{\rule[2cm]{\textwidth}{1.5pt}}
\vspace*{-2.4cm}
\end{center}

\lettrine[lines=2, findent = 1pt, nindent = 3pt, loversize=0.05]{I}{} t has become well-accepted nowadays that the digital revolution---spurred by the advent of the internet and social networking---has changed the way we interact, form bonds, and maintain our relationships to a degree not thought possible previously. Common public opinion has, as it always has with profound societal and technological changes, come to the conclusion that the changes brought about are to the detriment of people and society, and that our relationships are somehow less genuine and more isolated as a result of the technology we have at our fingertips. On the contrary, the truth is that our relationships have become more numerous, more intimate, and stronger in this new digital age, and all as a direct result of the new tools and digital technologies we have available to augment our lives.

It is sometimes easy to overlook the increased amount of intimacy digital technologies afford us; after all, we seem to spend less time engaging in face-to-face, personal interactions and more on communication through SMS texts, emails, and online instant messages.  However, what these modes of communication bring to the table in exchange for in-person engagement is a constant and deeper knowledge of the otherwise mundane details of other people's lives. This knowledge and engagement forms the bedrock of intimate relationships, and it allows participants to grow more comfortable with sharing more information and personal details with one another. In his article, \emph{ Brave New World of Digital Intimacy}\cite{nyt}, Clive Thompson calls this the phenomenon of ``ambient awareness,’’ and says that it can also ``lead to more real-life contact’’ by helping people meet each other without needing explicit planning---ad-hoc social networking. And when people do meet up in real life, they are already so intimately aware of the happenings in each other's lives that the conversation can be picked up as though it never stopped.

The idea that digital technology has improved the quantity and quality of our relationships is also supported by data and studies done by researchers in the field, and the results are in direct contradiction of the commonly assumed detriments of digital relationships. A study titled ``\emph{Social Networking Sites and Our Lives},’’ conducted by Dr. Keith Hampton of the University of Pennsylvania with the Pew Research Center \cite{prc}, has shown that internet users report more close relationships as well as greater diversity in these close relationships than non-internet users. Facebook users also claim to receive greater amounts of social support from members of their friend circles than others. And addressing the most prevalent concern of a lack of face-to-face interactions, the study showed that active internet users actually have a \emph{greater} number of them---perhaps due to how convenient these tools make it to meet-up without the time-consuming planning normally needed.

While relationships have been discussed in a more general sense till now, even the particular case of romantic relationships has benefitted from the use of digital tools, including increasing a sense of intimacy between partners. One noted effect is that digital and online communication releases some of the pressure from romantic interactions, allowing participants to interact in a casual way for as long as they feel comfortable; it is seen as normal and there is no need to necessarily proceed further in the relationship if the participants do not wish too.

For those already in relationships, digital tools help them stay constantly in touch and remain close, strengthening their sense of intimacy. One of the most important tools that aid in this is the mobile phone, where the facility of sending small, frequent, and inconsequential text messages allows couples to be constantly involved in each other’s lives no matter the distance between them. It has been found that online communication also helps couples maintain a more idealized image of their partners, which nevertheless helps their in-person interactions as well, especially in times of stress.

While it is of course true that some people use digital forms of communication so much that they effectively become dependent on it to an extent that all other modes of communication are lost, this is certainly not the norm. Further, people who use the tools and the power available constructively find the benefits are immense, and completely unrealizable in any other form.

From the numerous examples and explanations mentioned, it should be clear that digital technology provides many, many tools to people to aid them in creating and maintaining their bonds and relationships with others. These tools make it easier to find intimacy, and improves the quality of the relationship overall, regardless of whether it is a simple friendship, a romance, something in between, or something else entirely. Intimacy and relationships in the digital age have profoundly changed---but undoubtedly for the better.


\vspace{0.75cm}


\begin{center}
\pgfornament[width=7.5cm, color=spot]{85}
\end{center}

\newpage
\color{spot}
\begin{thebibliography}{99}
\color{black}
\bibitem{prc} Pew Research Center:
\emph{Social Networking Sites and Our Lives}. Last accessed September 13, 2013. URL: \url{pewinternet.org/Reports/2011/Technology-and-social-networks.aspx}

\bibitem{nyt} Clive Thompson: \emph{Brave New World of Digital Intimacy}, Last accessed September 13, 2013. URL: \url{www.nytimes.com/2008/09/07/magazine/07awareness-t.html}

\bibitem{blog} Elise Hu: \emph{On Digital Dating: Never Committing, And Never Breaking Up}. Last accessed September 13, 2013. URL: \url{www.npr.org/blogs/alltechconsidered/2013/07/22/204520278/on-digital-dating-never-committing-and-never-breaking-up}

\bibitem{upside} Rachel Pomerance: \emph{The Upside of Long-Distance Relationships}. Last accessed September 13, 2013. URL: \url{health.usnews.com/health-news/health-wellness/articles/2013/08/26/the-upside-of-long-distance-relationships}
\end{thebibliography}
\end{document}