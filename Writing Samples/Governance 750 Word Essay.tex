\documentclass[11pt,a4paper]{article}

\usepackage[hidelinks]{hyperref}
\usepackage[usenames, dvipsnames]{color}
\usepackage[object=vectorian]{pgfornament}
\usepackage{url}
\urlstyle{same}

\usepackage[quiet]{fontspec}
\defaultfontfeatures{Mapping=tex-text}
\setmainfont[Ligatures=TeX]{Sabon LT Std}

\usepackage{lettrine}

\definecolor{spot}{rgb}{0.6,0,0}

\usepackage[margin=1in]{geometry}

\usepackage[superscript,biblabel]{cite}
%\usepackage[square]{natbib}
%\setcitestyle{super}

\title{\vspace*{-0.35in}\Large{Writing Sample 2}\\\bigskip \huge{\color{spot}A Change in Education}\vspace{-0.2cm}}
\author{Milind Hegde}
\date{}

\pagestyle{empty}

\begin{document}
\fontsize{11.25}{18}\selectfont
\setlength{\parskip}{0.15cm plus4mm minus3mm}
\maketitle
\thispagestyle{empty}

\begin{center}
\textcolor{spot}{\rule[2cm]{\textwidth}{1.5pt}}
\vspace*{-2.4cm}
\end{center}

\lettrine[lines=2, findent = 1pt, nindent = 3pt, loversize=0.05]{W}{} hen I was nearing the end of 12th standard and had started considering which college I might end up spending the next four years of my life, I didn't face much confusion: I wanted to study math or pure science, and IISc, if I got it, was the clear best option. But for me, this lack of confusion (and for many others, the cause of confusion) was partly because of a strange feature of the landscape of India's universities: if you want to study science, you have to go to a university on list A; engineering, and choose from this second list; arts, and you get a third and perhaps also looks of disappointment and puzzlement. 

In short, there is no reputed institute of learning that embraces all the many forms of knowledge that people have invented and discovered over the centuries, regardless of where in our country you search. In my opinion, this is a terrible situation for youngsters who, like myself, were either not yet sure of what they wish to be doing with the rest of their lives, or were perhaps prematurely sure without having ever even heard of the many possibilities; like a child enthusiastically ordering the pictured sweet on the first page without knowing that ice cream is mentioned only on the second, and with no picture to boot.

What I would like to see change in India is for a reputed university, public or private, to arise and bring to one place all the many disciplines: arts, science, engineering, law, political science, business, medicine, philosophy. This is not a very tall demand, since many such well-known universities are present abroad. The students of such an institute would have the unique opportunity in India of being exposed to all the many disciplines and methods of thought and intellectual analysis that are present. They would be able to have conversations with other students from across the spectrum, and thereby gain new perspectives on issues and ideas, and realize that there is no “correct” and “incorrect” approaches to knowledge. And if something were to pique their interest, they would have the freedom to learn more about it from a guiding mentor and fellow students with similar interests. Today, a student that has the bad luck of becoming interested in a topic outside of his chosen-at-age-18 discipline does not have much option but to sit down, alone, with a book on the topic; conversation, discussion, and debate, vital components of learning in many areas, are simply not available.

The requirement for students to decide their preferred field of study at the end of 12th standard is an illogical one. At that stage of learning, a student does not truly have an idea of what any of the fields are or what it means to pursue one. What they have learnt thus far in their education has virtually no correlation with the actual nature of the field, as every college student soon realizes. It seems like a natural solution to allow students to make a more informed decision once they actually enter college and have an opportunity to learn about the disciplines at a more advanced and mature level, which is more indicative of the actual nature of the field. In short, it is the philosophy of our own undergraduate programme at IISc, but with a more intellectually broad purview.

It is surprising to me that the situation in India is as it is. It is in part an effect of the government's approach to educational institutions, wherein they have created specialized institutions for each discipline: IITs for engineering, IISc and IISERs for science, IIMs for management. Unfortunately, private colleges have also followed the same divided path. And, perhaps reflecting the general mindset of the public, the government has never made the effort to create a premier institute for the arts and humanities. (Though there is a possible exception to this with the inauguration of the Ashoka University in New Delhi in 2014, which aims to bring a multi-disciplinary and well-rounded education to students. But being such a recent initiative, it remains to be seen how successful it is. It is also not an initiative of the government, and the lack of any such initiative is perhaps most significant.)

This is, of course, quite a difficult idea to bring to reality, since it goes against everything in the current system and how it was built over the years. A particular difficulty is how it does not fit in with the general public's perception of college education and of the different disciplines, with the prominent example of the negative image of everything related to the arts. But if a strong initiative is made, by either the government or a private institute, change can finally begin.

To conclude, I would like to quote Daniel J. Boorstin, the late American historian, who said, ``Education is learning what you didn't even know you didn't know.'' It is a shame that by this definition, there is virtually no place in India where one can get a true education, which, simply put, needs to change. Give students a chance to learn anything and everything we wish to, and we will become well-rounded individuals, citizens, and humans. We are going to be the leaders and creators of tomorrow, after all, and having a knowledge and respect for all the many facets of the world will only help.





\vspace{0.75cm}


\begin{center}
\pgfornament[width=7.5cm, color=spot]{85}
\end{center}

\end{document}